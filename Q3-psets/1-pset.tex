\documentclass{article}
\usepackage{amsmath}
\usepackage{amssymb}
\usepackage{amsfonts}
\usepackage{mathrsfs}
\usepackage{float}
\usepackage{fancyhdr} % For custom headers and footers
\usepackage{geometry} % For page layout
\geometry{a4paper, margin=1in}
\usepackage{parskip}
\usepackage{comment}
\pagestyle{fancy} % Activate the fancyhdr package
\fancyhf{} % Clear all header and footer fields

\fancyhead[R]{\textbf{Deka, Karaminejad Ranjbar, Kumar, and Saggese \thepage}} % Right-aligned text

% Optional: Header rule customization
\renewcommand{\headrulewidth}{0.4pt} % Width of the header line


\begin{document}

\begin{titlepage}
    \centering
    \vspace*{1in} 
    {\Large Problem Set 1} \\[1.5cm] % we include anything else? 


    {\large \textbf{Rongmon Deka \\
    Kimia Karaminejad Ranjbar \\
    Ayush Kumar \\
    Allegra Saggese}} \\[0.75cm] % all our names
    {\Large Microeconomics 204C} \\[1.5cm]

    {\large Friday April 11, 2025} % Due date
    \\[2cm] 

\end{titlepage}

\section*{Problem 1} % Allegra 

\subsection*{(a)}

\[
C(\{x, y\}) = y, \quad C(\{x, z\}) = z, \quad C(\{y, z\}) = y, \quad C(\{x, y, z\}) = y
\]
\\
These choice sets result in the following preference relations $ y \succ x, z \succ x, y \succ z,$ and $y$ is preferred in a choice set containing all goods. Combining these preferences, we get a preference relation $y \succ z \succ x$ which is \textbf{consistent} for utility maximization. Given the budget sets, there is consistency of preferences, such that choice sets ($C(A)$), ($C(B)$), ($C(C)$) are within choice set ($C(D)$). We can say that $y$ is the best option, given $y \in A \subset D$, $y \in B \subset D$, and $y \in C \subset D$.  

\begin{itemize}
    \item \textbf{Necessity}:
        \begin{gather*}
        C(D) = argmax \{ u(y)\} \\
        y \in A\subset D \text{ and } y = C(D) \\ 
        C(D) = argmax \{ u(y)\} = y 
        \end{gather*}
    For any other variable, $x$, $U(x) < U(y) \forall y \in D$   because $B \subset D$, then $U(y) > U(x)$ for any $x \in B$, so $C(B) = y$. This also holds for $C(C)$.

    \item \textbf{Sufficiency}: Similarly, lets look at a set of alternatives, $X = \{x,y,z\}$ where $C(X) = y$, then $C(X$ \textbackslash $\{y\}) = z$ (second best choice). Another budget set, $B$, where $B = \{y,z\} = y$, allows us to infer $ y \succ z \succ x$. Set $A = (X$ \textbackslash $\{x,z\})$ then $C(A) = y$ and $y \in B \subset D$, so $C(B) = y$. This holds also for $C(C)$ such that $C = (X$ \textbackslash $\{x\})$. This infers a reduction in redundant alternatives does not affect the agents' choice of $y$. 
\end{itemize}

\subsection*{(b)}

\[
C(A) = C(\{x, y\}) = x, \quad C(B) = C(\{x, z\}) = z, \quad C(C) = C(\{y, z\}) = y, \quad C(D) = C(\{x, y, z\}) = y
\]

From these choice sets, we get the following results: 
\begin{gather*}
    x \succ y \quad z \succ x \quad y \succ z \quad y \succ \{x,z\} \\
    \implies y \succ z \succ x \succ y 
\end{gather*}
\\
The following inequality is inconsistent where, when we combine the results of each choice within the separate budget sets, we end up getting both $y \succ x$ and $ x \succ y$. This violates the transitivity property of preferences that are required to be rational, and thus achieve IIA. If choices are not IIA, \textbf{consistency with utility maximization does not exists}. Similarly, we can consider the redudancy of alternative choices. In one scenario, $C(A)$, $x \succ y$, while in the scenario $C(D)$, $y \succ x$. This violates the independence of irrelevant alternatives (IIA) as $A \subset D$. Inconsistency occurs because removing the redundant alternatives from $D$ does actually reflect a change in the agent's choice, as seen in $A$. 




\section*{Problem 2} % Kimia 


\subsection*{(a)}
\textbf{Consistent.}

In this case, consider:
\[
C(A) = \arg\min_{x \in A} \{ u(x) \}
\]

IIA requires that removing irrelevant alternatives does not affect the relative ranking. A non-minimum alternative is considered an irrelevant alternative. Removing such a term does not affect the alternative chosen.


\subsection*{(b)}
\textbf{Inconsistent.}

Based on the IIA definition, if \(x\) is the best in \(A\), then it is still the best in \(B \subseteq A\).

If \(x \in B \subseteq A\) and \(x = C(A)\), then \(x = C(B)\).

Let \(X = \{x_1, x_2, \ldots, x_n\}\) and it is ranked based on utility.

The median element is chosen.  
If the set has an odd number of elements,  
\[
\text{median} = x_{(n+1)/2}
\]

If the set has an even number of elements,  
the median is the average of the middle elements, so that
\[
\text{median} = \frac{x_{(n/2)} + x_{(n/2)+1}}{2}
\]

Consider the case of adding vs. removing a low-utility alternative.  
Removing the low-utility alternative shifts the element with the median utility, and the new set is a subset of the old set.   
Since a new median would be computed from removing an element, it is inconsistent with IIA.

\begin{itshape}
The choice function is not consistent with IIA.

Consider sets with an odd number of elements. Let $A = \{1, 2, 3, 4, 5\}$, where each element represents the utility given by the element. Then $C(A) = 3$. Define $B = \{1, 2, 3\}$, a strict subset of $A$ which includes $C(A) = 3$. But $C(B) = 2$, which violates IIA. Now define $B = \{3, 4 ,5\}$, another strict subset of A which includes $C(A) = 3$. Now $C(B) = 4$, which violates IIA.

Consider sets with an even number of elements. Let $A = \{1, 2, 3, 4\}$, where each element represents the utility given by the element. Then $C(A) = \{2, 3\}$. Define $B = \{1, 2, 3\}$, a strict subset of $A$ which includes $C(A) = \{2, 3\}$. But $C(B) = 2$, which violates IIA. Now define $B = \{2, 3, 4\}$, another strict subset of $A$ which includes $C(A) = \{2, 3\}$. $C(B) = 3$, which violates IIA.

Similar arguments show that the choice function is not consistent with IIA when the sets are continuous; for example, intervals in $\mathbb{R}$.
\end{itshape}



\section*{Problem 3} %Ayush
\subsection*{(a)}
Notice for $p_2=(2,1)$ and $m_2=24$ we have $x\cdot p_2=8+14=22\leq m_2=24$ thus $x$ is affordable for the new budget set, if $y=x$ then we have no contradiction. If $y\neq x$ then $y$ is revealed preferred to $x$, therefore $y$ must be unaffordable at $p_1=(1,1)$ and $m_1=18$ to be consistent with IIA and hence utility maximization. Hence, $y\cdot p_1=y_1+y_2>18$. Thus, the two conditions for being consistent are:
\begin{gather*}
   y\cdot p_1\leq m_1 \text{ and } y\cdot p_2\leq m_2  \text{ if } x=y \\
   y \cdot p_1>m_1  \text{ and } y\cdot p_2 \leq m_2 \text{ if } x\neq y
\end{gather*}

\subsection*{(b)}
Let $u(x)=x_1^\alpha x_2^\beta$ and $p=(2,1)$ and $m=24$ the maximization problem can be stated as:
\begin{gather*}
    \text{max } x_1^\alpha x_2^\beta \text{ s.t. } p\cdot x \leq m
\end{gather*}

the Lagrange thus can be written as:
\[
\mathcal{L}= x_1^\alpha x_2^\beta+\lambda(24-2x_1-x_2)
\]
FONC are:
\begin{gather}
    \alpha x_1^{\alpha-1}x_2^\beta-\lambda2=0 \\
    \beta x_1^\alpha x_2^{\beta-1}-\lambda=0 \\
    24-2x_1-x_2=0
\end{gather}
Multiplying $(2)$ by 2 and comparing with $(1)$ we have 
\begin{gather*}
    \alpha x_1^{\alpha-1}x_2^\beta=2\beta x_1^\alpha x_2^{\beta-1}
\end{gather*}
Note, optimal $x_1,x_2 \neq 0$ as if this were the case any positive deviation would yield greater utility thus, we have

\begin{gather*}
  \alpha x_2=2 \beta x_1 \text{ or } \frac{\alpha}{2\beta}x_2=x_1 
\end{gather*}

Substituting the above in $(3)$ we have 
\begin{gather*}
    24-\frac{\alpha}{\beta}x_2^*-x_2^*=0 \text{ hence, }\\
    x_2^*=\frac{24}{1+\frac{\alpha}{\beta}}=\frac{24\beta}{\beta+\alpha} \text{ and } x_1^*=\frac{12 \alpha}{\beta \{1+\frac{\alpha}{\beta}\}}=\frac{12\alpha}{\beta+\alpha}
\end{gather*}





\section*{Problem 4} % Kimia 

Given:
\[
C(\{x, y\}) = x\ \text{implies} \ x \ \text{is preferred to} \ y
\]
\[
C(\{y, z\}) = y\ \text{implies} \ y \ \text{is preferred to} \ z
\]

Let \( S = \{x, y, z\} \). 

By IIA,  
the choice from \( S \) must align most with the choices in all pairwise subsets of \( S \).

Evaluating the pairwise subsets of \( S \):

\[
C(\{x, y\}) = x \quad \text{implies} \quad y \notin C(S)
\]
\[
C(\{y, z\}) = y \quad \text{implies} \quad z \notin C(S)
\]


Considering set \( S \), \quad \( C(S) \neq y \) and \( C(S) \neq z \). \quad 
Thus \( C(S) = x \)

\vspace{1em}
By IIA, \quad \( C(\{x, z\}) \) must equal \( C(S) \) since \( \{x, z\} \in S \) \\
Thus \( C(\{x, z\}) = x \)











\section*{Problem 5} % rongmon
\subsection*{(i)}
\subsubsection*{i.} The choices available are $x = (8, 3)$ and $y = (4, 6)$. The reference points are 4 and 3.

\begin{align*}
    V(x) &= f(8 - 4) + f(3 - 3) = \sqrt{4} \\
    V(y) &= f(4 - 4) + f(6 - 3) = \sqrt{3}
\end{align*}

Anna prefers $x$ over $y$.

\subsubsection*{ii.} The choices available are $x = (8, 3)$, $y = (4, 6)$, and $d = (1, 5)$. The reference points are 1 and 3.

\begin{align*}
    V(x) &= f(8 - 1) + f(3 - 3) = \sqrt{7} \\
    V(y) &= f(4 - 1) + f(6 - 3) = 2\sqrt{3} = \sqrt{4}\sqrt{3} = \sqrt{12} 
\end{align*}

Anna prefers $y$ over $x$.

\subsubsection*{iii.} The choices available are $x = (8, 3)$, $y = (4, 6)$, and $z = (1, 7)$. The reference points are 1 and 3.

\begin{align*}
    V(x) &= f(8 - 1) + f(3 - 3) = \sqrt{7} \\
    V(y) &= f(4 - 1) + f(6 - 3) = 2\sqrt{3} = \sqrt{4}\sqrt{3} = \sqrt{12}
\end{align*}

Anna prefers $y$ over $x$.


\subsection*{(ii)} The choices available are $x = (8, 3)$, $y = (4, 6)$, and $z' = (t, 5)$, with $t > 0$.

Let $r_{1}$ and $r_{2}$ denote the reference points for the first and second attributes.

If $t < 4$, then $r_{1} = t$. If $t \geq 4$, then $r_{1} = 4$. We have $r_{2} = 3$.

If $t < 4$, then $r_{1} = t$ and $V(y) > V(x) \iff \sqrt{4 - t} + \sqrt{3} > \sqrt{8 - t}$. When $0 < t < 3.916$, $y$ is preferred to $x$. If $t \geq 4$, then $r_{1} = 4$ and $V(x) > V(y)$, as shown above.


\section*{Problem 6} %Ayush
\subsection*{(i)}
Let there be no status-quo bias then $r=(0,0)$ then $V_r(x)=f(7)+f(3)=7+3=10$ and similarly $V_r(y)=f(2)+f(7)=2+7=9$. Clearly $V_r(x)>V_r(y)$, therefore, Anna prefers $x$ over $y$ when there is no status quo.

\subsection*{(ii)}
Let $r=y=(2,7)$ then $V_r(y)=f(2-2)+f(7-7)=0$ and $V_r(x)=f(7-2)+f(3-7)=f(5)+f(-4)=5-8=-3$ thus $V_r(y)>V_r(x)$ hence, when $r=y$, $y$ is chosen over $x$.

\subsection*{(iii)}
Let $r=x=(7,3)$ then $V_r(x)=f(7-7)+f(3-3)=0$ and $V_r(y)=f(2-7)+f(7-3)=f(-5)+f(4)=-2\times5+4=-6$. As $V_r(x)>V_r(y)$ when $r=x$, $x$ is chosen over $y$.


\end{document}